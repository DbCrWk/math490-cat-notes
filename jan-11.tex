\documentclass[12pt]{article}

\usepackage[a4paper,margin=2cm]{geometry}
\usepackage{amsmath, amssymb, amsthm, amsfonts, tikz, algpseudocode}
\usepackage[plain]{algorithm}
\usepackage[framemethod=TikZ]{mdframed}
\definecolor{newblue}{rgb}{0.2,0.2,0.6}
\usepackage{caption}
\usepackage{graphicx}
\usepackage{float}
\usepackage{listings}
\usepackage{color}
\usepackage{xcolor}
\usepackage[colorlinks,allcolors=newblue]{hyperref}
\usepackage{listings}
\lstset{
    basicstyle=\scriptsize\ttfamily,
    commentstyle=\ttfamily\color{gray},
    numbers=left,
    numberstyle=\ttfamily\color{gray}\footnotesize,
    stepnumber=1,
    numbersep=5pt,
    backgroundcolor=\color{white},
    showspaces=false,
    showstringspaces=false,
    showtabs=false,
    frame=single,
    tabsize=2,
    captionpos=b,
    breaklines=true,
    breakatwhitespace=false,
    title=\ttfamily\lstname,
    escapeinside={},
    keywordstyle={},
    morekeywords={}
}

\input{config/lecture-header}
\input{config/math-boxes}

\definecolor{dfn-color}{rgb}{0,0.48,0.65}


\begin{document}
\lecture{Introduction}{January 11, 2018}{Dev Dabke}

\section{Logistics}\label{sec:logistics}

\begin{info}[Contact Information]{info:contact}
    \begin{itemize}
        \item Email address: \href{mailto:dv@math.duke.edu}{dv@math.duke.edu}
        \item Office Hours: To Be Determine, By Appointment
        \item Course Slack: \href{cattheory.slack.com}{cattheory.slack.com}
    \end{itemize}
\end{info}

\begin{info}[Course Structure]{info:structure}
    \begin{itemize}
        \item 50\% of grade: \( \approx 7 \) problem sets in \LaTeX
        \item 1 course project
        \item Some use of Haskell
    \end{itemize}
\end{info}

\section{Basic Definitions}\label{sec:definitions}

We will take the definition of a set for granted and we will assume the use of naive set theory.
Additionally, we assume the notion of a function (or map) and the standard notation for an application of a function \( f(\ldots) \).

\begin{dfn}[Relation]{dfn:relation}
    On some set \( X \), a \textcolor{dfn-color}{binary relation} is a function \( \textcolor{dfn-color}{\sim} \) such that
    \[
        \sim \, : X \times X \to \mathbb{B}
    \]
    where \( \mathbb{B} = \{ \top, \bot \} \).
    Where \( x_1, x_2 \in X \), we write \( x_1 \sim x_2 \) to mean \( \sim(x_1, x_2) = \top \).
    \\ \\
    In other words, a relation is a ``binary'' or ``2-ary'' function (i.e.\ it takes two inputs) and returns either a true or false value (also notated as a \( 1 \) or \( 0 \)).
\end{dfn}

\begin{dfn}[Pre-Order]{dfn:pre-order}
    A \textcolor{dfn-color}{pre-order} \( (P, \textcolor{dfn-color}{\leq}) \) is a set \( P \) with a binary relation \( \leq \) such that
    \begin{itemize}
        \item \( \forall x \in P, \, x \leq x \) (reflexivity)
        \item \( \forall x, y, z \in P, \, [x \leq y \text{ and } x \leq z] \implies x \leq z \) (transitivity)
    \end{itemize}
    We also say that \( x \) \textcolor{dfn-color}{precedes} \( y \) when \( x \leq y \).
    \\ \\
    If \( \leq \) satisfies the condition that
    \[
        \forall x, y \in P, \, [x \leq y \text{ and } y \leq x] \implies x = y
    \]
    then we say that \( P \) is a \textcolor{dfn-color}{partially ordered set}.
    \\ \\
    If \( \leq \) satisfies the condition that
    \[
        \forall x, y \in P, \, x \leq y \text{ or } y \leq x
    \]
    then we say that \( P \) is a \textcolor{dfn-color}{totally ordered set}.
\end{dfn}

\end{document}
